% Options for packages loaded elsewhere
\PassOptionsToPackage{unicode}{hyperref}
\PassOptionsToPackage{hyphens}{url}
%


\PassOptionsToPackage{table}{xcolor}

\documentclass[
  10pt,
  letterpaper,
]{article}

\usepackage{amsmath,amssymb}
\usepackage{iftex}
\ifPDFTeX
  \usepackage[T1]{fontenc}
  \usepackage[utf8]{inputenc}
  \usepackage{textcomp} % provide euro and other symbols
\else % if luatex or xetex
  \usepackage{unicode-math}
  \defaultfontfeatures{Scale=MatchLowercase}
  \defaultfontfeatures[\rmfamily]{Ligatures=TeX,Scale=1}
\fi
\usepackage{lmodern}
\ifPDFTeX\else  
    % xetex/luatex font selection
\fi
% Use upquote if available, for straight quotes in verbatim environments
\IfFileExists{upquote.sty}{\usepackage{upquote}}{}
\IfFileExists{microtype.sty}{% use microtype if available
  \usepackage[]{microtype}
  \UseMicrotypeSet[protrusion]{basicmath} % disable protrusion for tt fonts
}{}
\makeatletter
\@ifundefined{KOMAClassName}{% if non-KOMA class
  \IfFileExists{parskip.sty}{%
    \usepackage{parskip}
  }{% else
    \setlength{\parindent}{0pt}
    \setlength{\parskip}{6pt plus 2pt minus 1pt}}
}{% if KOMA class
  \KOMAoptions{parskip=half}}
\makeatother
\usepackage{xcolor}
\usepackage[top=0.85in,left=2.75in,footskip=0.75in]{geometry}
\setlength{\emergencystretch}{3em} % prevent overfull lines
\setcounter{secnumdepth}{-\maxdimen} % remove section numbering


\providecommand{\tightlist}{%
  \setlength{\itemsep}{0pt}\setlength{\parskip}{0pt}}\usepackage{longtable,booktabs,array}
\usepackage{calc} % for calculating minipage widths
% Correct order of tables after \paragraph or \subparagraph
\usepackage{etoolbox}
\makeatletter
\patchcmd\longtable{\par}{\if@noskipsec\mbox{}\fi\par}{}{}
\makeatother
% Allow footnotes in longtable head/foot
\IfFileExists{footnotehyper.sty}{\usepackage{footnotehyper}}{\usepackage{footnote}}
\makesavenoteenv{longtable}
\usepackage{graphicx}
\makeatletter
\def\maxwidth{\ifdim\Gin@nat@width>\linewidth\linewidth\else\Gin@nat@width\fi}
\def\maxheight{\ifdim\Gin@nat@height>\textheight\textheight\else\Gin@nat@height\fi}
\makeatother
% Scale images if necessary, so that they will not overflow the page
% margins by default, and it is still possible to overwrite the defaults
% using explicit options in \includegraphics[width, height, ...]{}
\setkeys{Gin}{width=\maxwidth,height=\maxheight,keepaspectratio}
% Set default figure placement to htbp
\makeatletter
\def\fps@figure{htbp}
\makeatother

% Use adjustwidth environment to exceed column width (see example table in text)
\usepackage{changepage}

% marvosym package for additional characters
\usepackage{marvosym}

% cite package, to clean up citations in the main text. Do not remove.
% Using natbib instead
% \usepackage{cite}

% Use nameref to cite supporting information files (see Supporting Information section for more info)
\usepackage{nameref,hyperref}

% line numbers
\usepackage[right]{lineno}

% ligatures disabled
\usepackage{microtype}
\DisableLigatures[f]{encoding = *, family = * }

% create "+" rule type for thick vertical lines
\newcolumntype{+}{!{\vrule width 2pt}}

% create \thickcline for thick horizontal lines of variable length
\newlength\savedwidth
\newcommand\thickcline[1]{%
  \noalign{\global\savedwidth\arrayrulewidth\global\arrayrulewidth 2pt}%
  \cline{#1}%
  \noalign{\vskip\arrayrulewidth}%
  \noalign{\global\arrayrulewidth\savedwidth}%
}

% \thickhline command for thick horizontal lines that span the table
\newcommand\thickhline{\noalign{\global\savedwidth\arrayrulewidth\global\arrayrulewidth 2pt}%
\hline
\noalign{\global\arrayrulewidth\savedwidth}}

% Text layout
\raggedright
\setlength{\parindent}{0.5cm}
\textwidth 5.25in 
\textheight 8.75in

% Bold the 'Figure #' in the caption and separate it from the title/caption with a period
% Captions will be left justified
\usepackage[aboveskip=1pt,labelfont=bf,labelsep=period,justification=raggedright,singlelinecheck=off]{caption}
\renewcommand{\figurename}{Fig}

% Remove brackets from numbering in List of References
\makeatletter
\renewcommand{\@biblabel}[1]{\quad#1.}
\makeatother

% Header and Footer with logo
\usepackage{lastpage,fancyhdr}
\usepackage{epstopdf}
%\pagestyle{myheadings}
\pagestyle{fancy}
\fancyhf{}
%\setlength{\headheight}{27.023pt}
%\lhead{\includegraphics[width=2.0in]{PLOS-submission.eps}}
\rfoot{\thepage/\pageref{LastPage}}
\renewcommand{\headrulewidth}{0pt}
\renewcommand{\footrule}{\hrule height 2pt \vspace{2mm}}
\fancyheadoffset[L]{2.25in}
\fancyfootoffset[L]{2.25in}
\lfoot{\today}
\usepackage{booktabs}
\usepackage{longtable}
\usepackage{array}
\usepackage{multirow}
\usepackage{wrapfig}
\usepackage{float}
\usepackage{colortbl}
\usepackage{pdflscape}
\usepackage{tabu}
\usepackage{threeparttable}
\usepackage{threeparttablex}
\usepackage[normalem]{ulem}
\usepackage{makecell}
\usepackage{xcolor}
% Remove comment for double spacing
% \usepackage{setspace}
% \doublespacing
\makeatletter
\makeatother
\makeatletter
\makeatother
\makeatletter
\@ifpackageloaded{caption}{}{\usepackage{caption}}
\AtBeginDocument{%
\ifdefined\contentsname
  \renewcommand*\contentsname{Table of contents}
\else
  \newcommand\contentsname{Table of contents}
\fi
\ifdefined\listfigurename
  \renewcommand*\listfigurename{List of Figures}
\else
  \newcommand\listfigurename{List of Figures}
\fi
\ifdefined\listtablename
  \renewcommand*\listtablename{List of Tables}
\else
  \newcommand\listtablename{List of Tables}
\fi
\ifdefined\figurename
  \renewcommand*\figurename{Figure}
\else
  \newcommand\figurename{Figure}
\fi
\ifdefined\tablename
  \renewcommand*\tablename{Table}
\else
  \newcommand\tablename{Table}
\fi
}
\@ifpackageloaded{float}{}{\usepackage{float}}
\floatstyle{ruled}
\@ifundefined{c@chapter}{\newfloat{codelisting}{h}{lop}}{\newfloat{codelisting}{h}{lop}[chapter]}
\floatname{codelisting}{Listing}
\newcommand*\listoflistings{\listof{codelisting}{List of Listings}}
\makeatother
\makeatletter
\@ifpackageloaded{caption}{}{\usepackage{caption}}
\@ifpackageloaded{subcaption}{}{\usepackage{subcaption}}
\makeatother
\makeatletter
\makeatother
\ifLuaTeX
  \usepackage{selnolig}  % disable illegal ligatures
\fi
\usepackage[numbers,square,comma]{natbib}
\bibliographystyle{plos2015}
\IfFileExists{bookmark.sty}{\usepackage{bookmark}}{\usepackage{hyperref}}
\IfFileExists{xurl.sty}{\usepackage{xurl}}{} % add URL line breaks if available
\urlstyle{same} % disable monospaced font for URLs
\hypersetup{
  pdftitle={Environmental Predictors Impact Microbial-based Postmortem Interval (PMI) Estimation Models Within Human Decomposition Soils},
  pdfauthor={Allison R. Mason; Hayden S. McKee-Zech; Dawnie W. Steadman; Jennifer M. DeBruyn},
  hidelinks,
  pdfcreator={LaTeX via pandoc}}



\begin{document}
\vspace*{0.2in}

% Title must be 250 characters or less.
\begin{flushleft}
{\Large
\textbf\newline{Environmental Predictors Impact Microbial-based
Postmortem Interval (PMI) Estimation Models Within Human Decomposition
Soils} % Please use "sentence case" for title and headings (capitalize only the first word in a title (or heading), the first word in a subtitle (or subheading), and any proper nouns).
}
\newline
\\
% Insert author names, affiliations and corresponding author email (do not include titles, positions, or degrees).
Allison R. Mason\textsuperscript{1}, Hayden S.
McKee-Zech\textsuperscript{2}, Dawnie W.
Steadman\textsuperscript{2}, Jennifer M. DeBruyn\textsuperscript{3,1*}
\\
\bigskip
\textbf{1} Department of Microbiology, University of
Tennessee-Knoxville, Knoxville, TN, USA, \\ \textbf{2} Department of
Anthropology, University of
Tennessee-Knoxville, Knoxville, TN, USA, \\ \textbf{3} Department of
Biosystems Engineering and Soil Science, University of
Tennessee-Knoxville, Knoxville, TN, USA, 
\bigskip

% Insert additional author notes using the symbols described below. Insert symbol callouts after author names as necessary.
% 
% Remove or comment out the author notes below if they aren't used.
%
% Primary Equal Contribution Note
\Yinyang These authors contributed equally to this work.

% Additional Equal Contribution Note
% Also use this double-dagger symbol for special authorship notes, such as senior authorship.
%\ddag These authors also contributed equally to this work.

% Current address notes
\textcurrency Current Address: Dept/Program/Center, Institution Name, City, State, Country % change symbol to "\textcurrency a" if more than one current address note
% \textcurrency b Insert second current address 
% \textcurrency c Insert third current address

% Deceased author note
\dag Deceased

% Group/Consortium Author Note
\textpilcrow Membership list can be found in the Acknowledgments
sections

% Use the asterisk to denote corresponding authorship and provide email address in note below.
* jdebruyn@utk.edu

\end{flushleft}

\section*{Abstract}
Microbial succession has been suggested to supplement established
postmortem interval (PMI) estimation methods for human remains. Due to
limitations of entomological and morphological PMI methods, microbes are
an intriguing target for forensic applications as they are present at
all stages of decomposition. Previous machine learning models from soil
necrobiome data have produced PMI error rates from two and a half to six
days; however, these models are built solely on amplicon sequencing of
biomarkers (e.g., 16S, 18S rRNA genes) and do not consider environmental
factors that influence the presence and abundance of microbial
decomposers. This study builds upon current research by evaluating the
inclusion of environmental data on microbial-based PMI estimates from
decomposition soil samples. Random forest regression models were built
to predict PMI using relative taxon abundances obtained from different
biological markers (bacterial 16S, fungal ITS, 16S-ITS combined) and
taxonomic levels (phylum, class, order, OTU), both with and without
environmental predictors (ambient temperature, soil pH, soil
conductivity, and enzyme activities) from 19 deceased human individuals
that decomposed on the soil surface (Tennessee, USA). Model performance
was evaluated by calculating the mean absolute error (MAE). MAE ranged
from 804 to 997 ADH across all models. 16S models outperformed ITS
models (p = 0.002), while combining 16S and ITS did not improve 16S
models alone (p = 0.47). Inclusion of environmental data in PMI
prediction models had varied effects on MAE depending on the biological
marker and taxonomic level conserved. Specifically, inclusion of the
measured environmental features reduced MAE for all ITS models, but
improved 16S models at higher taxonomic levels (phylum and class).
Overall, we demonstrate some level of predictability in soil microbial
succession during human decomposition, however error rates were high
when considering a large population of donors.


\linenumbers\hypertarget{introduction}{%
\section{Introduction}\label{introduction}}

Microbial communities undergo succession in response to disturbance
events \citep{shade_fundamentals_2012}. Vertebrate death and subsequent
decomposition represent one such event, where microbial community
composition is altered in response to nutrient deposition and altered
environmental conditions
\citep{howard_characterization_2010, pechal_potential_2014, metcalf_microbial_2013, cobaugh_functional_2015, javan_human_2016, mason_microbial_2023}.
Microbial succession has been studied in various carcass/cadaver
decomposition microhabitats, including internal organs
\citep{javan_thanatomicrobiome_2016, bell_sex-related_2018, can_distinctive_2014, lutz_effects_2020, li_potential_2021, hauther_estimating_2015, dong_succession_2019},
skin
\citep{pechal_potential_2014, metcalf_microbial_2013, hyde_initial_2015, pechal_large-scale_2018, johnson_machine_2016, metcalf_microbial_2016},
bone \citep{damann_potential_2015, emmons_postmortem_2022}, and soils
\citep{metcalf_microbial_2013, cobaugh_functional_2015, metcalf_microbial_2016, singh_temporal_2017, mason_body_2022}.
These studies suggest that these successional changes may be robust and
universal enough to be used to predict the postmortem interval (PMI), or
time elapsed since death (or beginning of decomposition). PMI
estimations can be important evidence for death investigations, allowing
law enforcement to establish a timeline of events
\citep{moffatt_improved_2016}. The work developing microbial-based PMI
models has been inspired by forensic entomology methods, which link
insect succession or development to PMI. Entomological PMI estimation
methods are widely used, but limited to forensic cases where insects are
present, the species is identifiable, and temperature data can be
collected. Unlike insects, microbes do not have a pre-colonization phase
where they must be exposed to, detect, and accept a carcass
\citep{tarone_is_2017, tomberlin_roadmap_2011}, as they are already
host-associated and present in the surrounding environment
\citep{mason_microbial_2023}. Together, this makes microbes advantageous
for developing a forensic application estimating time since death.

Thus far researchers have assessed microbial abundance-based PMI
prediction models in all major microhabitats and three different
mammalian species. This includes internal
\citep{liu_predicting_2020, hu_predicting_2021}, external/skin
\citep{pechal_potential_2014, johnson_machine_2016, metcalf_microbial_2016, belk_microbiome_2018, burcham_conserved_2024},
and soil \citep{belk_microbiome_2018, burcham_conserved_2024} microbial
communities during pig (\emph{Sus scrofa})
\citep{pechal_potential_2014}, mouse (\emph{Mus musculus})
\citep{metcalf_microbial_2016, liu_predicting_2020, belk_microbiome_2018},
and human (\emph{Homo sapiens})
\citep{johnson_machine_2016, metcalf_microbial_2016, hu_predicting_2021, belk_microbiome_2018, burcham_conserved_2024}
decomposition. These models suggest some level of predictability in
microbial succession; however, results differ based on the microhabitat,
taxonomic level considered, and algorithm used for model construction.
To date, most studies apply supervised machine learning algorithms to
taxonomic abundance data derived from amplicon sequencing of conserved
markers of a few taxa \citep{pechal_potential_2014} or whole microbial
communities
\citep{johnson_machine_2016, metcalf_microbial_2016, liu_predicting_2020, hu_predicting_2021, belk_microbiome_2018, burcham_conserved_2024}.
Within these studies, random forest regression is the most frequently
used supervised machine learning algorithm. These microbial PMI models
report error ranging from \textasciitilde15 hours for mouse intestine
samples to \textasciitilde58 hours in mouse brain samples
\citep{liu_predicting_2020}, up to 138 accumulated degree days for human
skin samples \citep{johnson_machine_2016}, and two to six days for soil
samples below decomposing mice and human \citep{belk_microbiome_2018}.

Of the three decomposition microhabitats in terrestrial decomposition
systems (\emph{i.e.}, internal, external, and soil), soils have received
the least attention. While there have been multiple studies assessing
internal and external/skin succession
\citep{pechal_potential_2014, johnson_machine_2016, metcalf_microbial_2016, liu_predicting_2020, hu_predicting_2021, belk_microbiome_2018},
only two in-depth studies have sought to predict PMI from soil microbial
community succession
\citep{belk_microbiome_2018, burcham_conserved_2024}. The first used a
mixed human and mouse dataset from n = 4 humans
\citep{belk_microbiome_2018}, while the other predicted PMI from a
dataset of n = 36 humans across geographical regions
\citep{burcham_conserved_2024}. While the current soil studies provide
foundational knowledge for estimating PMI from soil microbial
succession, it is unclear how inter-individual variation and potential
species differences may impact model performance, and thus what the
predictability would be across a large population of humans. Recent work
suggests differences in decomposition patterns may exist between species
\citep{dautartas_differential_2018, debruyn_comparative_2021} and even
within species, due to intrinsic carcass properties (\emph{e.g.}, body
composition) \citep{mason_body_2022}. Consequently, there is a need to
investigate the predictability of soil microbial succession within a
larger human dataset in order to assess applicability of these models
for forensic science. Additionally, current microbial-based PMI
estimation models are trained using relative abundance of microbial taxa
as model features. However, observed changes in soil environmental
parameters over time may also be used as indicators of decomposition
time. For example, soil electrical conductivity (correlates to
salinity), ammonium, and nitrate concentrations have been shown to
undergo predictable changes in human decomposition soils
\citep{aitkenhead-peterson_mapping_2012, fancher_evaluation_2017, taylor_high_2020}.
Thus, it is possible that the inclusion of environmental predictors
along with taxon relative abundance may help to improve model
estimation.

The goals of this study were to 1) determine the utility of soil
microbial communities for predicting decomposition time during human
decomposition across a large population; 2) determine which biological
marker (16S rRNA, Internal Transcribed Spacer (ITS), or both) and
taxonomic level (phylum, order, class, or OTU) results in the most
accurate model predictions; and 3) assess how inclusion of soil
environmental parameters (e.g., moisture, temperature, pH, conductivity,
and enzyme rates) as model features affects model accuracy. Our first
aim was to investigate microbial-based model performance across a large
human sample (n = 19) to validate previously developed models. Our
second aim was to evaluate model performance using different biological
markers, \emph{i.e.}, genes commonly used as sequencing targets.
Previous work reported that 16S rRNA gene-based models performed better
then 18S rRNA gene or ITS models in soil
\citep{belk_microbiome_2018, burcham_conserved_2024}. However, we
hypothesized that ITS-based models would be more accurate than 16S-based
models in a large dataset: In previous work, we observed that fungal
community composition between individuals became more similar over
decomposition time, whereas bacterial communities did not, suggesting
less noise and better predictability in the fungal communities
\citep{mason_body_2022}. We also aimed to assess which taxonomic
level(s) resulted in the greatest model accuracy. Based on previous
results \citep{belk_microbiome_2018, mason_body_2022}), we hypothesized
that higher taxonomic levels, such as phylum and class would provide
better PMI prediction. Our third aim was to probe the impact of
environmental features on model prediction. While no previous studies
have addressed this question, we hypothesized that inclusion of
environmental predictors known to change in decomposition soils would
help to improve PMI model predictions. Soil pH and conductivity are
known drivers of microbial community dynamics
\citep{lauber_pyrosequencing-based_2009, rath_linking_2019}, while
enzyme activities provide insight into functionality of the microbial
community, so we chose to evaluate these parameters. We addressed our
study aims using sequencing (16S rRNA and ITS2 amplicon) and soil
physicochemical data from 19 deceased human individuals
\citep{mason_body_2022} decomposed on the soil surface at the University
of Tennessee's Anthropology Research Facility (ARF) in Tennessee, USA.
Random forest regressions were applied to datasets with different
combinations of biological marker (16S only, ITS only, 16S and ITS) and
taxonomic level (phylum, class, order, OTU), both with and without
environmental predictors. Model performance was then compared by
calculating the mean absolute error (MAE) to determine the influence of
different combinations of features on PMI estimation.

\hypertarget{materials-and-methods}{%
\section{Materials and methods}\label{materials-and-methods}}

\hypertarget{study-design}{%
\subsection{Study design}\label{study-design}}

This work uses datasets generated from our previous study
\citep{mason_body_2022}, which revealed the influence of intrinsic, or
cadaver-related factors, on explaining variation in soil microbial
communities during human decomposition. Our current study uses these
datasets to assess the effects of environmental factors on
predictability of this succession to estimate PMI. Full experimental
details are reported in \citep{mason_body_2022}. Briefly, decomposition
of 19 deceased whole body human donors took place at the Anthropology
Research Facility (ARF), located at the University of Tennessee in
Knoxville, TN, USA (35\textdegree 56' 28'' N, 83\textdegree 56' 25'' W).
The ARF is a forested outdoor facility consisting of clay loam and
channery clay loam soils of the Coghill-Corryton complex (CcE)
\citep{mason_body_2022, keenan_spatial_2018, damann_potential_2015}.

Adult individuals with no open wounds or had not been autopsied were
chosen for this study, as this could alter microbial decomposers prior
to and during our study. Individuals were selected independent of
demographic, however all individuals self-identified as White and ranged
in age from 40 to 91 years \citep{mason_body_2022}. No living human
subjects were involved, and therefore this study was exempt from review
by the University of Tennessee Institutional Review Board. Individuals
were continuously placed supine unclothed on the soil surface between
February 2019 and March 2020. Hourly temperatures were recorded using
TinyTag temperature and humidity loggers (Gemini Data Loggers, UK) until
un-enrollment at the end of active decomposition, characterized by
collapse of the abdomen and cessation of fluid leaking from the trunk
\citep{mason_body_2022, megyesi_using_2005}. Accumulated degree hours
(ADH) were calculated using hourly temperature readings: 0 ADH was
defined as time of placement within ARF, and a baseline temperature of
10\textdegree C was used for ADH calculations to keep our results
comparable with entomology-based methods \citep{byrd_development_2001}.

\hypertarget{soil-sampling-and-analysis}{%
\subsection{Soil sampling and
analysis}\label{soil-sampling-and-analysis}}

As previously described in \citep{mason_body_2022}, five cm soil cores
were collected from the decomposition-impacted area surrounding each
individual at predetermined accumulated degree hour (ADH) intervals
until the end of active decomposition. ADH intervals included prior to
placement, 100, 250, 500, 750, and 1000 ADH, and thereafter at 500 ADH
intervals until un-enrollment. For each respective sample, cores were
homogenized and debris (\emph{e.g.}, roots, insect larvae, rocks, etc.)
removed by hand. Soil samples were flash frozen in liquid nitrogen and
stored at -80\textdegree C prior to DNA extraction. Soil pH and
electrical conductivity (EC) were measured from soil slurries using an
Orion Star\texttrademark  A329 pH/ISE/Conductivity/Dissolved Oxygen
portable multiparameter meter (ThermoFisher), as described in
\citep{mason_body_2022} and gravimetric was measured in duplicate by
oven drying 2 to 3 g soil aliquots at 105\textdegree C for 72 hours.
Enzyme activities of \(\beta\)-glucosidase (BG),
N-acetyl-\(\beta\)-D-glucosaminidase (NAG), leucine amino peptidase
(LAP), and alkaline phosphatase (PHOS) were measured according to a
modified procedure by Bell et al.~(2013)
\citep{mason_body_2022, bell_high-throughput_2013}.

\hypertarget{dna-extraction-sequencing-and-amplicon-sequence-analysis}{%
\subsection{DNA extraction, sequencing, and amplicon sequence
analysis}\label{dna-extraction-sequencing-and-amplicon-sequence-analysis}}

DNA extracts were obtained as described in \citep{mason_body_2022}.
Briefly, 0.25 g of soil was extracted with the DNeasy Powerlyzer
PowerSoil kit (QIAGEN Inc.) following manufacturer's instructions with
modifications for our soil texture (clay loam) and condition (high
organic content). Specifically, soils were homogenized under parameters
suggested for high organic soils (2,500 RPM for 45 s). DNA concentration
was determined using a fluorometric assay
(Quant-iT\texttrademark  PicoGreen\textregistered  dsDNA Assay Kit
(Invitrogen)) with a total volume of 200 \(\mu\)l and 1 \(\mu\)l of DNA.
All DNA extracts were sent to the University of Tennessee Sequencing
Core Facility (Knoxville, TN) for 16S rRNA and ITS2 region amplicon
sequencing on the Illumina MiSeq platform (2 x 150 bp). The primer set
515F \citep{parada_every_2016} /806R \citep{apprill_minor_2015} was used
to amplify the V4 region of the 16S rRNA gene, while the ITS2 region in
fungi was amplified using primers described previously
\citep{mason_body_2022, cregger_populus_2018}. All raw sequences have
been deposited in the National Center for Biotechnology Information's
Sequence Read Archive under the BioProject
\href{https://www.ncbi.nlm.nih.gov/bioproject/?term=PRJNA817528}{PRJNA817528}.

Raw sequences were processed in Mothur \citep{schloss_introducing_2009}
(v.1.43.0) to cluster into 97\% similarity operational taxonomic units
(OTUs) and generate OTU count tables for both 16S and ITS datasets as
described in \citep{mason_body_2022}, though in this study all OTUs were
retained (singletons and doubletons were not removed). Additionally,
control samples and samples greater than 5000 ADH were removed in R.
Samples were cut off at 5000 ADH to capture the linear response of soil
parameters and account for variation in decomposition timeframes between
individuals observed in \citep{mason_body_2022}. This resulted in 113
samples from 19 individuals for model construction.

\hypertarget{machine-learning-models}{%
\subsection{Machine learning models}\label{machine-learning-models}}

Read counts were total sum scaled (TSS) by determining the relative
abundance of each OTU and normalizing to a standard library size (10,000
for all samples) using phyloseq \citep{mcmurdie_phyloseq_2013}
(v1.32.0). This allowed for comparison of reads across samples and
between biomarkers. We also removed OTUs with less than 10 reads across
all samples in TSS normalized count tables to reduce noise in the
datasets. 16S and ITS TSS read count tables were generated at the
phylum, order, and class levels by summing the corresponding OTU table
at each respective taxonomic level and then applying the TSS
normalization as described above. One of our goals was to compare
predictability of bacterial (16S only), fungal (ITS only), or both
(16S-ITS) communities; therefore, after TSS normalization, 16S-ITS
combined datasets for each taxonomic level (phylum, order, class, OTU)
were generated by merging respective 16S and ITS TSS count tables. As a
result, 12 datasets were created and used for random forest models.

We chose to apply random forest regression to datasets to predict PMI in
ADH. This kept our study similar to those previously conducted in
decomposition soil \citep{belk_microbiome_2018}, while also assessing
predictability of soil microbial succession during human decomposition
in our geographical region (Knoxville, TN). Model construction was
completed in R using the Ranger \citep{wright_ranger_2017} (v0.13.1)
package. First, samples were assigned to testing or training datasets.
This was completed by randomly assigning 6 donors (\textasciitilde1/3)
to the test set, while the remaining 13 were grouped into the training
set. This approach was conducted following Belk et al.~(2018)
\citep{belk_microbiome_2018}, to ensure that all samples from a single
individual were in either the testing or training set, respectively, to
prevent overfitting.

Next, random forest regressions were applied to microbial taxa TSS
normalized count tables in Ranger. First, random forest model parameters
node size (3, 5, 7, 9) and sample size (0.55, 0.632, 0.70, 0.80) were
hyper-tuned by comparing models with different combinations of the
parameters listed. The optimal model was chosen by assessing the
out-of-bag mean square error (OOB MSE) of each model and choosing the
set of parameters with the lowest OOB MSE. The optimum model for each
biomarker and taxonomic level was assessed by calculating the OOB MSE of
the model and the root mean square error (RMSE) and mean absolute error
(MAE) for predictions of the testing set in 100 runs of the optimum
model. RMSE and MAE were calculated using rmse() and mae() functions
from the R package Metrics (v 0.1.4). This process was repeated for
models including measured environmental parameters, with values for
ambient temperature (\textdegree C), pH, electrical conductivity (EC),
moisture, \(\beta\)-glucosidase (BG) activity,
N-acetyl-\(\beta\)-D-glucosaminidase (NAG) activity, leucine amino
peptidase (LAP) activity, and alkaline phosphatase (PHOS) activity
included as model features (Table~\ref{tbl-modelparams}). For all
environmental parameters, aside from temperature, log response ratio
normalized values \citep{mason_body_2022, risch_effects_2020} were used
to account for natural seasonal differences in these parameters. The top
25 most influential model features were extracted from each optimum
model to assess taxa/environmental factors influencing model
predictions. To evaluate the potential differences in model predication
between biomarkers and taxonomic levels, linear regression was applied
to the average MAE values (mean of 100 runs per model). Variation in MAE
due to treatment variables was assessed with ANOVA, while differences
between treatment groups were determined with post-hoc t-tests in R.
Code for generating all feature tables and random forest model
development can be found at
https://github.com/jdebruyn/TOX-microbiology.

\hypertarget{tbl-modelparams}{}
\begin{longtable}[]{@{}
  >{\raggedright\arraybackslash}p{(\columnwidth - 2\tabcolsep) * \real{0.2172}}
  >{\raggedright\arraybackslash}p{(\columnwidth - 2\tabcolsep) * \real{0.7828}}@{}}
\caption{\label{tbl-modelparams}Overview of variables used to construct
models. OTU = Operational taxonomic unit.}\tabularnewline
\toprule\noalign{}
\begin{minipage}[b]{\linewidth}\raggedright
Type of data
\end{minipage} & \begin{minipage}[b]{\linewidth}\raggedright
Predictor variables tested
\end{minipage} \\
\midrule\noalign{}
\endfirsthead
\toprule\noalign{}
\begin{minipage}[b]{\linewidth}\raggedright
Type of data
\end{minipage} & \begin{minipage}[b]{\linewidth}\raggedright
Predictor variables tested
\end{minipage} \\
\midrule\noalign{}
\endhead
\bottomrule\noalign{}
\endlastfoot
Bacterial and fungal taxa (OTUs) relative abundances & 16S OTUs, ITS
OTUs, both 16S and ITS OTUs \\
Phylogenetic level & Phylum, order, class, OTU \\
Environmental parameters & Ambient temperature, soil electrical
coductivity, pH, moisture, enzyme activities (\(\beta\)-glucosidase,
N-acetyl-\(\beta\)-D-glucosaminidase, leucine amino peptidase, and
alkaline phosphatase) \\
\end{longtable}

\hypertarget{results}{%
\section{Results}\label{results}}

\hypertarget{soil-environmental-parameters}{%
\subsection{Soil environmental
parameters}\label{soil-environmental-parameters}}

We previously reported how the measured soil parameters were altered in
response to human decomposition \citep{mason_body_2022}. In summary,
soil EC increased with progression of decomposition in soils surrounding
all decomposing individuals. Soil pH was variable between individuals,
with pH increasing (n = 5 individuals), decreasing (n = 12), or
displaying minimal change relative to the controls (n = 2)
\citep{mason_body_2022}. Extracellular enzyme activities were also
variable between individuals, however general trends included increased
NAG and PHOS over time. BG and LAP were variable over time;LAP activity
correlated to soil pH \citep{mason_body_2022}.

\hypertarget{general-model-statistics}{%
\subsection{General model statistics}\label{general-model-statistics}}

In total, 24 models were built in R. Twelve of the models contained
environmental features and the other twelve did not. The number of taxa
included as features in models without environmental data are reported
in Table~\ref{tbl-numfeatures}. Bacterial (16S) and fungal (ITS)
features ranged from 35 to 5195 and 16 to 2219, respectively, depending
on taxonomic level. For all models, MAE ranged from 804.18 to 996.8 ADH
(Fig~\ref{fig1}). Across all variables considered, the best performing
model was the 16S phylum level model with environmental predictors (MAE
804.18) (\nameref{s1-table}). In contrast, the worst performing model
was the ITS phylum level without environmental data (MAE 996.8)
(\nameref{s1-table}). Predictability, assessed by the linear
relationship between predicted and observed values, for the training
(Fig~\ref{fig2}A-C) and testing (Fig~\ref{fig2}D-F) datasets for the
best 16S only, ITS only, and 16S-ITS models are shown in Fig~\ref{fig2}.
\(\mathrm{R}^2\) for these models ranged from 0.881 to 0.937, however
these values were reduced when making predictions for the testing
dataset (\(\mathrm{r}^2\) = 0.581 - 0.719).

\hypertarget{tbl-numfeatures}{}
\begin{longtable}[]{@{}lrrr@{}}
\caption{\label{tbl-numfeatures}Number of microbial taxonomic features
provided as input for random forest regression investigated in this
study. 16S-ITS microbial features are the sum of 16S and ITS
features.}\tabularnewline
\toprule\noalign{}
& 16S Features & 16S-ITS Features & ITS Features \\
\midrule\noalign{}
\endfirsthead
\toprule\noalign{}
& 16S Features & 16S-ITS Features & ITS Features \\
\midrule\noalign{}
\endhead
\bottomrule\noalign{}
\endlastfoot
OTU & 5195 & 7414 & 2219 \\
Order & 264 & 411 & 147 \\
Class & 111 & 177 & 66 \\
Phylum & 35 & 51 & 16 \\
\end{longtable}

% Place figure captions after the first paragraph in which they are cited.
\begin{figure}[!h]
\caption{{\bf Mean absolute error (MAE) from 100 iterations of each respective model against the testing dataset.}
Data are reported by biological marker (column), while color compares models with (gold) and without (gray) environmental predictors. Error bars are the standard error of MAE values across all 100 iterations for each respective model.}
\label{fig1}
\end{figure}

\begin{figure}[!h]
\caption{{\bf Model predictions for the training set (A-C) and testing set (D-F) for the top performing model for each biological marker as determined by the lowest MAE.}
For each biological marker, top models included the 16S phylum + environmental data (A, D) 16-ITS order (B, E), and ITS order + environmental data (C, F). Predictability of each model is greater for the training set (A-C) compared to the testing set (D-F). Red (training set) and blue (testing set) lines show the best fit linear relationship and shading indicted the 95\% confidence interval between actual PMI, in ADH, and predicted ADH within each respective dataset.}
\label{fig2}
\end{figure}

\hypertarget{model-comparison-biological-marker}{%
\subsection{Model comparison: biological
marker}\label{model-comparison-biological-marker}}

Ability of random forest regressions to predict ADH varied depending on
the biological marker used to build models (ANOVA F = 9.655, \emph{p} =
0.001). ITS models were generally less accurate in predicting ADH
compared to 16S or 16S-ITS models independent of taxonomic level and
environmental data (Fig~\ref{fig3}). ITS models ranged in MAE from
872.16 to 996.8 ADH, with a mean MAE of 909.59 ADH. Post-hoc t-tests
show that ITS models, in general, had higher MAE than both 16S (t-test
\emph{p} = 0.002) and 16S-ITS (\emph{p} = 0.006) models. ITS models
represented seven of the 10 worst performing models. In comparison, 16S
and 16S-ITS models performed similarly (\emph{p} = 0.47). 16S models
ranged in MAE from 804.18 to 889.81 ADH, with an average MAE of 841.73
ADH, while 16S-ITS models ranged in MAE from 812.35 to 890.94 ADH
(average MAE = 852.82 ADH). This can also be observed among the best and
worst performing models, where no ITS-only model was in the top 10 best
performing models and combined 16S-ITS models were dispersed among the
best and worst models. For example, the 16S-ITS order level model
without environmental data had the third lowest MAE (MAE 812.35)
overall, but also the 16S-ITS class level model with environmental data
had the sixth highest MAE (MAE 890.94).

% Place figure captions after the first paragraph in which they are cited.
\begin{figure}[!h]
\caption{{\bf Mean absolute error (MAE) varies as a result of biological marker (16S, 16S-ITS, or ITS) used for model construction.}
Average MAE is the result of 100 iterations of the 24 respective models against the testing set. Reported p-values are the result of post-hoc t-tests.}
\label{fig3}
\end{figure}

\hypertarget{model-comparison-taxonomic-level}{%
\subsection{Model comparison: taxonomic
level}\label{model-comparison-taxonomic-level}}

Some variation was observed in MAE due to taxonomic level considered for
model development, however these differences were not significant (ANOVA
F = 1.538; \emph{p} = 0.24). When considering the potential influence of
taxonomic level within biomarkers, no significant difference in MAE by
taxonomic level was observed for 16S (\emph{p} = 0.0.141) or ITS
(\emph{p} = 0.609) models, while 16S-ITS models was significant
(\emph{p} = 0.048), driven by a difference in MAE between order and
class level models (Fig~\ref{fig4}). While most results were not
significant, some trends were observed. First, order level models had
the lowest MAE for all three biological markers assessed. This was also
observed in Fig~\ref{fig1}, where order level models had the lowest MAE
for all models without environmental data. Trends for the other
taxonomic levels varied depending on the biological marker in
consideration. Phylum and class level models performed similarly within
16S models, with OTU models generating the highest MAE. Within 16S-ITS
models, class and OTU level models performed similarly, displaying the
first and second highest MAE, respectively. For ITS models, phylum level
models had the highest MAE, followed by OTU and class.

% Place figure captions after the first paragraph in which they are cited.
\begin{figure}[!h]
\caption{{\bf Mean absolute error (MAE) does not vary as a result of taxonomic level (color) used for model construction for any of the biological markers assessed (column).}
Mean MAE is the result of 100 iterations of the 24 respective models against the testing set. Order level model (green) generally had the lowest MAE, compared to phylum (purple), class (blue), and OTU (yellow) models. ANOVA p-values are the result of linear models comparing mean MAE to taxonomic level.}
\label{fig4}
\end{figure}

\hypertarget{model-comparison-environmental-parameters}{%
\subsection{Model comparison: environmental
parameters}\label{model-comparison-environmental-parameters}}

Overall, inclusion of environmental parameters in random forest models
to predict ADH from soil microbial taxa impacted model accuracy. The
direction of effect (\emph{i.e.}, increase or decrease in MAE) was
dependent on biological marker and taxonomic level considered
(Fig~\ref{fig1}). For ITS models, inclusion of environmental factors
reduced MAE irrespective of the taxonomic level. This reduction was most
pronounced for the phylum level model, in which MAE was reduced by
116.007 ADH. In models containing 16S sequencing data (16S and 16S-ITS),
effect of environmental features differed by taxonomic level.
Specifically, for 16S models, phylum, class, and order level models
performed better and OTU level models performed worse when environmental
data was included. This was similar for the combined 16S-ITS datasets at
the phylum and OTU levels; however, class and order level models
performed worse (\emph{i.e.}, increased MAE) when environmental factors
were included.

\hypertarget{model-features-top-models}{%
\subsection{Model features: top
models}\label{model-features-top-models}}

In addition to assessing the predictability of different random forest
models, we also looked at important model features to observe which taxa
and/or environmental parameters impacted model performance. Here we
highlight the top 25 features of best performing 16S (phylum +
environmental), 16S-ITS (order), and ITS (order + environmental) models,
determined by lowest MAE within the test set. Both 16S and ITS best
models included environmental predictors, while the combined 16S-ITS
model did not (Fig~\ref{fig5}A-C). For the 16S phylum model with
environmental data, the most important model feature, as assessed by
decrease in MSE, was the phylum \emph{Firmicutes}. The remaining
important features included soil electrical conductivity (EC),
\emph{Acidobacteria}, \emph{Epsilonbacteraeota}, and
\emph{Proteobacteria}, respectively. Other features of interest for this
model included \emph{Nitrospirae}, leucine aminopeptidase activity, pH,
and soil moisture (Fig~\ref{fig5}A). For the ITS order model with
environmental predictors, the most important model features were
\emph{Pleosporales}, soil EC, Unclassified fungi, \emph{Rhizophydiales},
Unclassified \emph{Glomeromycota}, Unclassified \emph{Basidiomycota},
and \emph{Auriculariales} (Fig~\ref{fig5}B). In this model, no other
environmental parameters were among the top 25 important features. Other
top taxonomic features of interest included \emph{Saccharomycetales} and
Unclassified \emph{Sordariomycetes}, as their members are present in the
human mycobiome and feces, respectively
\citep{blackwell_fungi_2004, taylor_ascomycota_2015}. The best
performing 16S-ITS model was the order level model without environmental
features. Top features for this model were the bacterial order
\emph{Lactobacillales} and the fungal order \emph{Pleosporales}.
Bacterial orders \emph{Bacteroidales}, \emph{Cardiobacteriales}, and
\emph{Clostridiales} were third, fourth, and eleventh most important
features, respectively (Fig~\ref{fig5}C). The fungal order
\emph{Saccharomycetales} was also observed in the top 25.

Relative abundance of anaerobic bacterial taxa identified in random
forest models, including \emph{Firmicutes} (Fig~\ref{fig6}A) and
\emph{Clostridiales} (Fig~\ref{fig6}C), increased as decomposition
progressed. In contrast, relative abundance of the aerobic nitrifying
organisms of the phylum \emph{Nitrospirae} decreased (Fig~\ref{fig6}B).
Relative abundance of the bacterial order \emph{Cardiobacteriales}
(Fig~\ref{fig6}D) and fungal order \emph{Pleosporales} (Fig~\ref{fig7}),
identified in the mixed 16S-ITS order model, increased and decreased,
respectively, over time.

% Place figure captions after the first paragraph in which they are cited.
\begin{figure}[!h]
\caption{{\bf Top 25 model features determined by variance of responses in Ranger.}
For each biological marker, top models included 16S phylum + environmental data (A), ITS order + environmental data (B), and 16S-ITS order (C). Bar color denotes whether the feature is a 16S taxon (green), ITS taxon (orange), or environmental feature (purple).}
\label{fig5}
\end{figure}

\begin{figure}[!h]
\caption{{\bf Relative abundance of select bacterial taxa identified as important features in random forest regressions.}
Relative abundance of the phyla Firmicutes (A) and Nitrospirae (B) and orders Clostridiales (C) and Cardiobacteriales (D) change over time, here accumulated degree hours (ADH), within decomposition-impacted soils. Trends for each of the 19 individuals (named “TOX\#\#\#”) are delineated by color.}
\label{fig6}
\end{figure}

\begin{figure}[!h]
\caption{{\bf Relative abundance of the fungal order Pleosporales over time, here accumulated degree hours (ADH).}
Trends for each of the 19 individuals (named “TOX\#\#\#”) are delineated by color.}
\label{fig7}
\end{figure}

\hypertarget{discussion}{%
\section{Discussion}\label{discussion}}

The goal of this work was to assess the influence of biological marker,
taxonomic level, and environmental parameters on model prediction of PMI
from soil microbial communities. Model analysis revealed differences
between model accuracy due to the biological marker, taxonomic level,
and environmental parameters considered for model construction. Overall,
models generally performed poorly, ranging in MAE from 804.18 to 996.8
ADH. In East Tennessee, these error rates would correspond to roughly
2.5 to 3.5 days in July and greater than 28 days in February, based on
average seasonal temperatures for the region. Therefore, error rates in
the summer would be comparable to those previously reported for soil
microbial communities (two to six days) \citep{belk_microbiome_2018} but
would be substantially higher if considering decomposition during cooler
seasons. Further, considering our total decomposition time of 5000 ADH,
errors of 804.18 to 996.8 ADH equates to 15.9\% to 19.9\% of the total
decomposition time. The wide error range when including a large number
of subjects across multiple seasons suggests microbiome-based models may
have low accuracy, particularly when considering individuals across the
cline of human variation and through multiple seasons. Specifically, the
decomposition systems were influenced by the starting resource, which is
dictated by human variation at both the genetic and environmental
levels. As a result, intrinsic factors have the capacity to alter both
decomposer communities and decomposition rate, and therefore decomposer
communities, leading to variation that can impact future models.

One important source of variation in our study was the different rates
of decomposition between individuals. While we attempted to correct for
differences due to thermal energy input by using accumulated degree
hours (ADH), there was still variability in terms of the morphological
stage for a given ADH. For example, 5000 ADH represented the end of
active decomposition for individual 009, but only about 25\% of active
decomposition for individual 010. Additionally, this time-period did not
include decomposition past active decay for any individual in our study,
including advanced decomposition or sustained mummification or
skeletonization, which could further impact model accuracy. Both
individuals (009 and 010) were placed within the facility in the summer,
experiencing the same local environmental conditions and potential for
insect and scavenger communities, suggesting there are additional
factors leading to variation in microbial communities within
decomposition-impacted soils. These may include additional environmental
parameters not considered in our models, and/or intrinsic differences
between the individuals themselves (\emph{e.g.}, age, weight,
medications, medical conditions etc.) that directly or indirectly
impacted microbial communities through interactions with other
decomposers (insects and scavengers). Moving forward, we will need to
employ a strategy to combine antemortem and environmental data in order
to investigate which factors help improve model predictions.

\hypertarget{influence-of-diversity-and-taxa-succession-on-pmi-estimations}{%
\subsection{Influence of diversity and taxa succession on PMI
estimations}\label{influence-of-diversity-and-taxa-succession-on-pmi-estimations}}

The trends we observed in model MAE between different biological
markers, taxonomic levels, and inclusion of environmental data may be
partly explained by differences in diversity between bacterial and
fungal (16S vs.~ITS) communities and the number of taxa (\emph{i.e.},
features), ultimately impacting resolution for predicting PMI. Overall,
Chao1 richness and Inverse Simpson diversity were 10 and 15 times lower,
respectively, in fungal communities compared to bacterial communities
\citep{mason_body_2022}. This translated to differences in the total
number of model features for 16S and ITS models: 16S models had roughly
1.7 to 2.3 times more features, depending on the taxonomic level
considered. As a result, more features, or taxa, in the dataset with
relationships to time (\emph{i.e.}, progression of decomposition) may
help to distinguish between timepoints to improve model predictability.

In our previous work, we observed that the fungal community composition
became more similar as decomposition progressed, with only a few taxa
driving fungal successional patterns \citep{mason_body_2022}. This was
also observed in Fu et al.~(2019) \citep{fu_fungal_2019}, in which only
a few taxa (\emph{e.g.}, \emph{Ascomycota} sp., \emph{Yarrowia
lipolitica}, etc.) displayed relationships with PMI. While we
hypothesized that ITS-based models would be more accurate than 16S-based
models because of these studies, our results revealed that 16S models
generally outperformed ITS-based models and combining 16S and ITS did
not improve 16S models alone. This result coincides with those reported
by Belk et al.~(2018) \citep{belk_microbiome_2018}, in which 16S models
had lower error than ITS or 18S models. The reduced number of taxa
observed in fungal communities, in combination with relatively few taxa
changing in abundance, may explain why ITS models had higher MAE than
16S models. This may also explain why combining 16S and ITS datasets,
which would increase overall diversity, did not outperform either marker
alone. With only a few fungal taxa displaying changes over time, their
inclusion may not have added additional resolution to the bacterial
model.

Diversity differences between bacterial and fungal communities may also
drive some of the trends observed between taxonomic levels and with or
without environmental factors. In this study, order level models
performed best for all biological markers when not considering
environmental features. This contrasts with findings by Belk at
al.~(2018) \citep{belk_microbiome_2018}, where lower error was reported
for phylum and class level models. This may be linked to a balance
between taxonomic resolution and noise for this timeperiod of
decomposition. In our study, OTU level models displayed the highest MAE
for all biological markers when not considering environmental data,
which corresponds with previous decomposition studies reporting
increased inter-individual variation at lower taxonomic levels
\citep{metcalf_microbial_2016, mason_body_2022, taylor_soil_2024}. This
may explain why OTU level models displayed the highest MAE for all
biological markers when not considering environmental data. Within
decomopsition studies, microbial taxonomic succession has mostly been
characterized at higher taxonomic levels, at which general patterns are
more similar between individuals. However, aggregating microbial
abundances at coarse taxonomic levels, such as phyla and class,
inherently reduces data dimensionality. It is possible that this
decrease in features, in conjunction with trends in taxon abundance over
time, reduces the ability of the random forest regression to resolve
timepoints at the highest taxonomic levels.

This balance between diversity and features with resolution over time
may also explain the effect of environmental features effect on model
MAE. We hypothesized that inclusion of environmental parameters would
improve all model predictions, by combining soil chemical and microbial
successional patterns. While inclusion of environmental predictors
improved some models, it decreased performance of others. This effect
appears to be linked to biological marker and taxonomic level considered
for model creation. Specifically, inclusion of environmental parameters
into the lower diversity fungal models added features that helped to
improve overall resolution to predict PMI. In contrast, inclusion of
environmental data may have added additional noise to high diversity
bacterial datasets at lower taxonomic levels, overall leading to
decreased model performance.

\hypertarget{model-features}{%
\subsection{Model features}\label{model-features}}

In addition to assessing model performance, we also investigated model
features for each top performing 16S, 16S-ITS, and ITS model as
determined by lowest MAE. This included the 16S phylum model with
environmental data, the 16S-ITS order level model, and the ITS order
level model with environmental data. Top model features for 16S phylum
level models included taxa observed in previous human and animal
decomposition studies, such as the bacterial phyla \emph{Firmicutes},
\emph{Acidobacteria}, and \emph{Proteobacteria}
\citep{cobaugh_functional_2015, metcalf_microbial_2016}. In our study,
\emph{Firmicutes} and \emph{Nitrospirae} were shown to decrease as
decomposition progressed, while \emph{Acidobacteria} decreased and
\emph{Proteobacteria} remained consistent. These changes seem to be
linked to differences in metabolism and environmental changes that occur
when decomposition products are released into the surrounding soil. For
example, it has been suggested heterotrophic microbial activity
responding to the pulse of decomposition products results in depletion
of soil oxygen \citep{taylor_soil_2024, keenan_mortality_2018}. This
would impact the presence of anaerobic gut and soil taxa. While we did
not measure soil oxygen in this study, soil respiration was increased in
these soils, and so oxygen depletion is to be expected
\citep{mason_body_2022}. The increased presence of taxa containing
facultative and obligate anaerobic members, \emph{Firmicutes} and
\emph{Clostridiales} in phylum and class models, respectively, and
decrease in \emph{Nitrospirae}, containing nitrifying bacteria that
oxidize nitrogen under aerobic conditions, support this hypothesis.
Increases in \emph{Firmicutes} and \emph{Clostridiales} follow
successional trends observed in internal (\emph{e.g.}, organs) microbial
communities. Specifically, increased relative abundance of
\emph{Clostridium} has been termed the ``Clostridium Effect'' by Javan
et al.~(2017) \citep{javan_cadaver_2017} and observed in various organs
\citep{javan_human_2016, javan_cadaver_2017} and the rectum
\citep{debruyn_postmortem_2017} postmortem. Multiple studies, including
this current work, have observed increased relative abundance in
\emph{Firmicutes} and \emph{Clostridiales} in soils following deposition
of decomposition fluid
\citep{cobaugh_functional_2015, mason_microbial_2023, singh_temporal_2018, keenan_microbial_2023},
suggesting some of these organisms may be host-derived. Decreased
presence of \emph{Acidobacteria} in decomposition-impacted soils is
likely linked to their oligotrophic characteristics in response to high
nutrient deposition \citep{cobaugh_functional_2015, fierer_toward_2007}.
Succession of these taxa and other taxa past 5000 ADH and the potential
implications for PMI models is unclear.

Order level models also revealed some information about soil microbial
succession during human decomposition. The 16S-ITS order level model had
the lowest MAE among all 16S-ITS models. Within this model, important
taxa were a combination of 16S and ITS features present in respective
models. Among the top bacterial features, \emph{Lactobacillales},
\emph{Bacteroidales}, and \emph{Clostridiales} were all shown to display
general increases as decomposition progressed. This is consistent with
previous literature \citep{cobaugh_functional_2015}. One interesting
find was \emph{Cardiobacteriales} as the fifth most important model
feature. \emph{Cardiobacteriales} is a bacterial order of gram-negative
rods, whose members are generally capable of fermentation of various
sugars \citep{garrity_order_2007}. Within this order, only the genus
\emph{Ignatzschineria} was identified based on the SILVA non-redundant
database (v132) \citep{quast_silva_2013}. This genus has been identified
in previous outdoor decomposition studies focusing on gut
\citep{debruyn_postmortem_2017}, skin \citep{hyde_initial_2015}, and
soil \citep{cobaugh_functional_2015} microbial communities.
\emph{Ignatzschineria} are associated with insect species
\citep{toth_proposal_2007, gupta_ignatzschineria_2011} and first appear
in the soil during release of fluid. We previously observed this taxon
in bacterial decomposition fluid communities \citep{mason_body_2022},
further suggesting these organisms come from the body. Their association
with insects highlights the potential for decomposer insect and
scavenger activity to impact microbial succession during decomposition
and suggests that PMI estimation models specific to decomposition
setting (indoor or outdoor) may be required.

Within the ITS order level models, the fungal order \emph{Pleosporales}
was among the most influential taxon for PMI estimation.
\emph{Pleosporales}, a member of \emph{Ascomycota} fungi, decreased as
decomposition progressed. This was similar to observations by Fu et
al.~(2019) \citep{fu_fungal_2019}, where \emph{Pleosporales} sp. was
shown to be associated with non-decomposition soils by LEfSe (linear
discriminant analysis effect size). \emph{Pleosporales} are often
associated with plants, found as endophytes, epiphytes, and the
rhizosphere {[}60{]}. Reduced relative abundance of these fungi in
decomposition soils is interesting considering \emph{Pleosporales} have
been shown to positively respond to nitrogen amendments
\citep{lowell_comparative_2001, she_resource_2018}. Their response to
decomposition products may suggest sensitivity to highly concentrated
nitrogen amendments and/or other soil changes, such as osmotic stress in
response to high EC, or intolerance to hypoxia typically observed in
decomposition soils.

Of the environmental predictors assessed, electrical conductivity (EC)
appeared to be most influential. EC was recorded as the top and second
most important feature for the 16S phylum and ITS order models with
environmental data, respectively. This is likely due to patterns in soil
EC being more consistent between individuals over time. Specifically, EC
was shown to increase within decomposition soil over time for all 19
individuals. Increases in EC observed in decomposition soils has been
shown to positively correlate with increased ammonium concentrations
\citep{keenan_mortality_2018, keenan_spatial_2019}, suggesting ammonium
would also be a valuable predictor of microbial community dynamics. The
other measured environmental parameters (pH, enzyme rates) were not
identified as a top predictive features in the models. This is likely
because these parameters were more variable both over time as well as
between individuals, displaying both increases and decreases in response
to human decomposition \citep{mason_body_2022}. While we did not
consider all possible environmental parameters in this study, these
results suggest that feature selection may help to identify relevant
environmental parameters for model construction.

\hypertarget{limitations-and-considerations}{%
\subsection{Limitations and
considerations}\label{limitations-and-considerations}}

While there are intriguing investigations suggesting that microbial
succession could be used to predict PMI, validation is critical prior to
forensic application. Variation between decomposition studies, including
vertebrate species observed, and experimental design, along with small
sample sizes have limited model development to date. Additionally, most
decomposition studies focus on bloat and active decay stages, when
decomposers are most active in degrading soft tissues
\citep{payne_summer_1965, carter_cadaver_2007}. While informative for
initial compositional shifts, this timeframe does not allow us to assess
for how long these communities may be impacted or if they return to
pre-decomposition conditions \citep{shade_fundamentals_2012}. This study
starts to address factors that influence PMI estimations from soil
microbial succession using a larger set of human individuals, however
many foundational questions remain. Below we discuss multiple areas to
be expanded upon with future investigation.

First, we did not include all possible environmental and soil data as
model predictors, nor account for interactions with other decomposer
communities. Other factors, such as respiration rates, oxygen
concentration, ammonium, nitrate, dissolved organic carbon and nitrogen,
sulfur, among others may be relevant features for models predicting PMI
within the soil environment they have been shown to change during
decomposition
\citep{cobaugh_functional_2015, metcalf_microbial_2016, mason_body_2022, debruyn_comparative_2021, aitkenhead-peterson_mapping_2012, fancher_evaluation_2017, keenan_spatial_2018, keenan_mortality_2018, quaggiotto_dynamic_2019, szelecz_soil_2018, macdonald_carrion_2014, anderson_dynamics_2013, meyer_seasonal_2013, benninger_biochemical_2008, vass_time_1992, taylor_soil_2023}
and have the capacity to structure microbial communities. In addition,
it is possible that changes in soil parameters during human
decomposition differ based on region due to soil type and climatic
differences impacting decomposition progression or presence of microbial
taxa \citep{carter_temperature_2008}, as well as the insect and
scavenging species present across ecosystems. Lines of inquiry should
include, but are not limited to, regional and seasonal (both within and
between regions) soil microbial successional patterns in response to
carcass decomposition and microbial-insect interactions, including
effects of local insect species on microbial community dynamics. For
example, \emph{Chrysomya megacephala}, an invasive fly species that has
a proclivity for feces, has only recently been documented colonizing
human remains in Tennessee, USA \citep{owings_first_2021}. This species
carries up to 10 times the pathogenic bacterial load compared to the
house fly, \emph{Musca domestica}, potentially introducing microbes that
could alter the progression of decomposition and result in different
microbial community succession between regions with and without this fly
species \citep{chaiwong_blow_2014}.

Second, we chose to implement the random forest regression algorithm as
it is not as sensitive to non-linear data and has high interpretability
compared to other forms of supervised and unsupervised machine learning
algorithms \citep{ghannam_machine_2021}. This allowed us to assess
prediction of PMI and identify taxa and environmental features that
correlate with PMI, as well as kept our results similar to previous
decomposition studies within the soil environment
\citep{belk_microbiome_2018}. However, recent studies have compared
multiple machine learning algorithms in other decomposition
microhabitats (\emph{i.e.}, skin, organs), showing variation both
between internal organs \citep{liu_predicting_2020} and within the same
habitat \citep{johnson_machine_2016}. Both Liu et al.~(2020)
\citep{liu_predicting_2020} and Johnson et al.~(2016)
\citep{johnson_machine_2016} observed other machine learning algorithms
performed better than random forest in higher diversity microhabitats
such as the skin and caecum. As the soil environment is among the most
diverse microbial habitat on the planet, it is necessary to assess
different machine learning approaches when predicting PMI within this
microhabitat \citep{metcalf_estimating_2019}.

Third, total PMI (5000 ADH) considered for model construction will
likely impact the performance and applicability of these models
\citep{metcalf_estimating_2019}. Here we showed that order level models
had the lowest MAE in models that do not include environmental features.
This contrasts with findings by Belk at al.~(2018)
\citep{belk_microbiome_2018} and our hypothesis, in which we expected
lower error for phylum and class level models, suggesting differences
between studies, such as region, sampling strategy, number of
individuals, species, study timeframe or intrinsic differences between
donor populations may impact model performance. For example, our study
went through 5000 ADH, while Belk et al.~(2018)
\citep{belk_microbiome_2018} presented data up to 25 days. In our study,
5000 ADH corresponded to 13 to 115 days depending on the individual and
time of year. Additionally, Belk et al.~(2018)
\citep{belk_microbiome_2018} observed decreased model error when only
using data points from the first 25 days of decomposition compared to
the first 50 days. Therefore, future work is needed to determine the
optimum PMI for microbial-based PMI models.

Fourth, we chose to use operational taxonomic units (OTUs) and ADH
calculated with a baseline of 10\textdegree C, as opposed to amplicon
sequence variants (ASVs) and/or ADH with a baseline of 0\textdegree C or
4\textdegree C. The recent application of denoising methods to generate
ASVs has become popular in microbial studies using amplicon sequencing.
However, we chose to cluster sequences into OTUs to reduce
dimensionality in our raw dataset and the probability of splitting
single genomes across multiple ASVs
\citep{patrick_d_schloss_amplicon_2021}. While Glassman and Martiny
(2021) \citep{glassman_broadscale_2018} observed similar results for
alpha and beta diversity from OTUs and ASVs in leaf litter communities,
other studies have shown differences in diversity when comparing the two
methods \citep{chiarello_ranking_2022}. Thus, it is unclear if using
OTUs or ASVs will impact machine learning algorithms such as random
forest to predict PMI. Future work should investigate differences in PMI
estimations from models constructed with OTUs as well as ASVs.

Lastly, we chose to use a baseline of 10\textdegree C, which is commonly
used for entomological methods due to the developmental threshold of
regional (east Tennessee) insects \citep{byrd_development_2001}.
However, other decomposition studies within the soil environment have
also used 0\textdegree C or 4\textdegree C as thresholds for ADH or
accumulated degree hour (ADD) calculations. These differences may impact
PMI estimates; however, no one has addressed effects of different
thresholds for ADH/ADD calculation on PMI estimates from microbial
successional patterns within the soil. Therefore, a comprehensive
comparison between different thermal energy unit (\emph{i.e.}, ADH, ADD
and baseline) calculations is necessary.

\hypertarget{conclusion}{%
\section{Conclusion}\label{conclusion}}

This study aimed to assess microbial abundance-based prediction of PMI
from soil microbial communities. We compared models with different
biological markers, taxonomic levels, and presence/absence of
environmental variables to expand upon previous estimations of PMI from
machine learning algorithms. From this large dataset of 19 individuals
across multiple seasons, we observed higher error rates and decreased
model precision compared to previously published models based on small
datasets. Our results show that 16S and 16S-ITS models performed
similarly and outperformed ITS models. Further, order level models have
the lowest MAE when not considering environmental parameters. We also
show that the addition of other factors, such as environmental
parameters, have the potential to impact PMI estimations. We observed
some level of predictability in soil microbial succession, however high
error rates were seen across 19 individuals and across seasons. While
our study size is one of the largest to date, it was demographically
limited, and we certainly did not capture all antemortem conditions
which could influence decomposition rates. Together this means
microbial-based PMI models would need considerable validation and
refinement across a diverse population and geographical regions prior to
implementing in a forensic context.

\hypertarget{supporting-information}{%
\section{Supporting information}\label{supporting-information}}

\paragraph*{S1 Table.}
\label{s1-table}
{\textbf{Summary statics for all random forest models. Values are means
for 100 runs of each model. OOB MSE = out-of-bag mean squared error,
RMSE = root mean squared error, MAE = mean absolute error, OTU =
Operational taxonomic unit.}}

\hypertarget{acknowledgments}{%
\section{Acknowledgments}\label{acknowledgments}}

The authors would like to thank the donors and donor families, whose
generosity of body donation made this research possible. We would like
to acknowledge multiple individuals who helped with design and execution
of the field study that supported this work. We would like to
acknowledge Mary Davis, Sarah Schwing, Erin Patrick, and Thomas Delgado
for preparation of donors prior to study. We would also like to thank
Sarah Schwing, Erin Patrick, Katharina Hoeland, and Thomas Delgado for
their help collecting daily observations of donors and soil samples used
in this study.

This project was supported by the National Institute of Justice (Award
No.~DOJ-NIJ-2018-DU-BX-018) to DWS and JMD. The opinions, findings, and
conclusions or recommendations expressed in this manuscript are those of
the authors and do not necessarily reflect those of the Department of
Justice.


\nolinenumbers
  \bibliography{bibliography.bib}

\end{document}
